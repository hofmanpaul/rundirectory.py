\documentclass[11pt, a4paper]{article}
\usepackage{graphicx} %for pictures
\usepackage{apacite} %For APA style references
\def\sym#1{\ifmmode^{#1}\else\(^{#1}\)\fi} % This is for the stata tables. By putting it here rather than before each table we don't need to repeat it every time

\graphicspath{ {../02_Output/} } %where pictures are

\title{Example rundirectory.py}
\author{Paul Hofman}
\begin{document}

\maketitle

This document gives an example how rundirectory.py can pull together analysis and cleaning done in R, Stata and Python.

For example, in R we compared balance across treatments:

\begin{figure}[!htbp]\centering 
	\includegraphics[width=\textwidth]{balance.jpg}
\end{figure}

It looks like we have balance! 

Next, let's check if there is  a treatment effect: we analyse this in Stata.

\begin{table}[!htbp]\centering
    \begin{tabular}{l*{2}{r}}
    	\hline\hline
    	\input{../02_Output/results.tex}
    	\hline\hline
  	\end{tabular}
\end{table}	

Doesn't look like a treatment effect, that's a bummer but expected as \citeA{hofman2018} found the same.

\bibliography{paperrefs}
\bibliographystyle{apacite}

\end{document}